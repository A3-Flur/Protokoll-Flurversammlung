\documentclass[10pt,a4paper,final]{article}
\usepackage[utf8]{inputenc}
\usepackage[german]{babel}
\usepackage{amsmath}
\usepackage{amsfonts}
\usepackage{amssymb}
\usepackage{makeidx}
\usepackage{graphicx}
\usepackage[left=2cm,right=2cm,top=2cm,bottom=2cm]{geometry}
\usepackage{color}
%\author{Leandro Marcelo Torres}
\title{Protokoll der Flurversammlung}
\date{\today}

\begin{document}
\maketitle %make a comment
%\thispagestyle{empty}
\cleardoublepage
\tableofcontents
\newpage

\section{Anwesenheitsliste}
\begin{table}
	\centering
		\begin{tabular}{|c|c|c|c|}
		\hline
			Zimmernummer	&	Mitbewohner	&	Funktion	& Anwesend	\\
			\hline
			A301	&	Roanak	&		-	&	Ja\\
			\hline
			A302	&	Jurij	&	-	&	Ja\\
			\hline
					A303	&	Leandro	&	-	&	Ja\\
		\hline
				A304	&	Iwey	&	-	& Nein\\
		\hline
				A305	&	Jakob	& Waschmaschinenminister	&	Ja\\
		\hline
				A306	&	Vali	&	Flurstellvertreter	&	Ja\\
		\hline
				A307	&	Svanje	&	-	& Spät erschienen\\
		\hline
				A308	&	Konstantin	&	-	& Ja\\
		\hline
				A309	&	Andrés	& Getränkenminister& Ja		\\
		\hline
				A310	&	Rafael	& Flursprecher&		Ja\\
		\hline
				A311	&	Pablo	&		-	& Nein\\
		\hline
				A312	&	Andreas	&	-	& Ja\\
		\hline
				A313	&	Felix	&- & Ja\\
		\hline
				A314	&	Aline	&	-& Ja\\
		\hline
					A315	&	Max		& Flurkasse& Ja\\
			\hline
			
		\end{tabular}
	\caption{Anwesenheitsliste}
	\label{tab:Anwesenheitsliste}
\end{table}

\section{Begrüßung und Feststellung der Beschlussfähigkeit}
Rafael eröffnet die Sitzung und stellt die Beschlussfähigkeit fest.
\section{Ernennung des neuen Flursprechers}
Rafael stellt sich selber zur Wahl vor und wird wieder gewählt
\section{Genehmigung des A3-Codes}
Der A3-Code wird per Akklamation genehmigt.
\subsection{Modifizierungen}
\subsubsection{Ernennung eines Flurrates}
Es wird ein Flurrat erstellt. Der Flurrat besteht aus:
\begin{itemize}
	\item Flursprecher
	\item Flurstellvertreter
	\item Waschmaschinenminister
	\item Getränkenminister
	\item Finanzreferent
\end{itemize}
Der Flurrat entscheidet ob eine Strafe erteilt wird oder nicht. Wird ein Mitbewohner (MW1) von einem anderen Mitbewohner (MW2) gestört (bsp. Lautstärke), so hat MW1 MW2 darauf hinzuweisen. Wird das störende Verhalten von MW2 nicht unterlassen, so kann sich er MW1 an den Flurrat wenden. Der Flurrat entscheidet über die Erteilung und das Maß der Strafe.
\subsubsection{Einführung einer 25Euro-Kaution für die Waschmaschinennutzung und einer 25Euro-Kaution für den Getränkenverbrauch}
\subsubsection{Kaufen von Putzmittel. Mülldienst}
\subsubsection{Gäste}
Gäste auf die Klonutzung hinweisen. Sitzen, stehen, Toilettenburst usw. Klopapier nicht unsachgemäß gebrauchen.
Wer Mülldienst hat, muss auch alle Putzmittel kaufen und die Biosäcke bei den Hausmeistern holen.
Jeder Mitbewohner muss beim Eiziehen die oben genannten Kautionen bei den jeweiligen Ministern bezahlen.
\subsubsection{Wäscherständer}
Die Wäscherstände dürfen aus Brandschutzgründen nicht im Flur stehen. Nichts desto trotz stehen die auf dem Flur, da die Kleidung irgendwo aufhängen muss. Wenn die Wäscherständer leer sind müssen sind sofort vom Flur geräumt werden.
\subsubsection{herumliegende Sachen}
Alles, was nicht mit einem Namen beschriftet ist Eigentum vom Flur und jeder kann sich daraus bedienen oder die Sache in den Müll schmeißen, falls es stört.
\subsubsection{Duschen}
Wasserhahn fest zudrehen.
\section{Festlegung des Termins für die nächste Flurversammlung}
Ein Termin für die nächste Flurversammlung wurde nicht festgelegt.
\section{Festlegung des Termins für den nächsten Flurputz}
Der nächste Flurputz findet am 07.06.2015 statt.
	\subsection{Überprüfung, ob alle für den Flurputz benötigten Putz- und Desinfektionsmittel vorhanden sind}
	Dies wurde nicht gemacht. Man muss es auf jeden Fall vor dem Flurputz machen.
\section{Festlegung des Termins für die nächste Tour de Chambre}
Die nächste Tour de Chambre wird voraussichtlich am 30.05.2015 nach dem Neueinzügleressen stattfinden. Das Neueinzügleressen wird um 20 uhr stattfinden. 
\section{Flurbeitrag}
Die Mitbewohner, die den Flurbeitrag noch nicht bezahlt haben sind:
\section{Waschmaschine}
\subsection{Bericht des Waschmaschinenministers}
\subsection{Benennung des neuen Waschmaschinenministers}
\section{Getränke im Flur}
\subsection{Bericht des Bierministers}
\subsection{Benennung des neuen Bierministers}
\section{Feedback der Mitbewohner}
\section{Anschaffungen}
\begin{enumerate}
	\item Klostein, Jakob, Preis nicht bekannt.
	\item Decke für die Mikrowelle, Felix, Preis nicht bekannt.
	\item Griff für die Außenseite der Balkontür, Felix, Preis nicht bekannt.
	\item Moskitonetz, Felix und Andreas, Preis nicht bekannt.
	\item Bierbank, wird bei Gelegenheit geklaut, Preis nicht bekannt.
\end{enumerate}

%\section{Teilnahme an Festen}
%Hier wird darüber entschieden, ob sich der Flur mit einem Flurstand an einem Fest beteiligen möchte.
%\subsection{Brainstorming}
%\begin{enumerate}
	%\item Idee
	%\item Idee
%\end{enumerate}
%\subsection{Wahl der Verantwortlichen}
%\begin{itemize}
	%\item Funktion, Mitbewohner, Zimmernummer
%\end{itemize}
\section{Sonstiges}
\subsection{Jalousien}
Rafael fragt am HP wie die Anschaffungen von Jalousien in anderen Häusern des Hadikos erfolgt.
\subsection{Flurbeitrag}
Jurij merkt an, der Flurbeitrag ist zu hoch. Die anderen MW sehen es nicht so. Der Flurbeitrag bleibt unverändert.
\subsection{Mülldienst}
Rafael erneuert die Schilder für die unterschiedlichen Mülltonnen.
\subsection{Klammer erneut beschriften}
Die Klammer, mit denen die MW zu den von denen hinterlassenen dreckigen Geschirr zugeordnet werden kann müssen neu beschriftet werden. Rafael übernimmt die Aufgabe.
\subsection{Abtauen des Gefrierschrankes}
Max übernimmt die Aufgabe
\subsection{Duschen}
Jakob macht ein Schild mit der Beschriftung "`Wasserhahn richtig zudrehen"'
\subsection{Sauberkeit}
\subsubsection{Küche}
Herd nach jeder Nutzung sauber machen. Geschirr abtrocknen. Mikrowelle sauber machen (Rafael macht das). Der Grillmodus (bzw. Grillknopf) der Mikrowelle muss beschriftet werden. Waschbecken nach dem Spülen von Essensresten sauber machen.
\subsubsection{Wohnzimmer}
Tisch nach dem Essen nass abwischen.
\subsubsection{Toilette}
Toilettenburst benutzen.
\subsubsection{Balkon}
Balkon nach der Nutzung sauber machen. Beim Grillen die Balkontür zumachen. Bierkästen wegräumen. Aschenbecher reinigen.

\end{document}